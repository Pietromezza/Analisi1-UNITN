\chapter{Dimostrazione per induzione}
\section{Definizione di Pn}
$P_n$ è una proposizione (o affermazione) che dipende da n.

\section{Principio di induzione debole}
Sia $P_n$ una proposizione definita $\forall\ n>n_0$;  $n,n_0 \in \mathbb{N}$.
\begin{equation}
\begin{cases}
\textit{i. }P_{n_0}\ \text{è vero}\\
\textit{ii. }\forall\ n>n_0 \text{, supposta vera } P_n \text{ è vera anche } P_{n+1}
\end{cases} \iff P_n \text{ è vera}
\end{equation}
L'affermazione \textit{ii.} è chiamata \textbf{scatto induttivo}.

\section{Principio di induzione forte}
Sia $P_n$ una proposizione definita $\forall\ n>n_0$;  $n,n_0 \in \mathbb{N}$.
\begin{equation}
\begin{cases}
\textit{i. }P_{n_0}\ \text{è vero}\\
\textit{ii. }\forall\ n>n_0 \text{, supposte vere } P_{n_0},P_{n_0+1},...,P_n \text{ è vera anche } P_{n+1}
\end{cases} \iff P_n \text{ è vera}
\end{equation}

\section{Osservazione}
Varianti del problema possono porre come incognita anche $n_0$. In questo caso, conviene dimostrare inizialmente \textit{ii.} e poi cercare, fra i risultati ottenuti, il più piccolo numero $n_0 \in \mathbb{N}$ che soddisfi anche \textit{i.} .