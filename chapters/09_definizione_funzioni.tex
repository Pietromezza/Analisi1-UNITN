\chapter{Funzioni}
\section{Definizione}
Dati due insiemi $\mathbb{X},\mathbb{Y}$ qualsiasi, una funzione di dominio $\mathbb{X}$ e valori in $\mathbb{Y}$ (=codominio) è una qualsiasi legge che ad ogni elemento di $\mathbb{X}$ associa un (e \textbf{uno solo}) elemento di $\mathbb{Y}$.
\begin{equation}
\begin{gathered}
f: \mathbb{X} \rightarrow \mathbb{Y}\\
x \rightarrow y=f(x)
\end{gathered}
\end{equation}

\section{Funzioni particolari}
\subsection{Funzione costante}
\begin{equation*}
\begin{gathered}
\bar{y} \in \mathbb{Y}, f: \mathbb{X} \rightarrow \mathbb{Y}\\
f(x)=\bar{y}\ \forall\ x \in \mathbb{X}
\end{gathered}
\end{equation*}
\subsection{Funzione identità}
\begin{equation*}
\begin{gathered}
I_x: \mathbb{X} \rightarrow \mathbb{X}\\
I_x(x)=x\ \forall\ x \in \mathbb{X}
\end{gathered}
\end{equation*}
\subsection{Funzione restrizione}
\begin{equation*}
\begin{gathered}
f: \mathbb{X} \rightarrow \mathbb{Y}; \mathbb{A} \subseteq \mathbb{Y}\\
f\restriction_\mathbb{A} =f(x)\ \forall\ x \in \mathbb{A}\\
f\restriction_\mathbb{A}: \mathbb{A} \rightarrow \mathbb{Y};\ x \rightarrow f(x)
\end{gathered}
\end{equation*}

\section{Insieme Immagine}
Sia $f: \mathbb{X} \rightarrow \mathbb{Y}$ e $\mathbb{A} \subseteq \mathbb{Y}$, diremo $\mathbb{A}$ tramite $f$ l'insieme $f(\mathbb{A}) = \{f(x) | x \in \mathbb{A}\}$.\\
Se $\mathbb{A} = \mathbb{X}$, $f(\mathbb{X})$ è semplicemente detta Immagine di $f$; $f(\mathbb{X}) = Imf$

\section{Iniettività, Suriettività e Biettività}
\subsection{Iniettività}
Una funzione $f: \mathbb{X} \rightarrow \mathbb{Y}$ si dice \textbf{iniettiva} se:
\begin{equation*}
\forall x_1,x_2 \in \mathbb{X}, x_1 \neq x_2, f(x_1) \neq f(x_2)
\end{equation*} 
Analogamente, il concetto di iniettività può essere definito come:
\begin{equation*}
f(x_1) = f(x_2) \iff x_1=x_2
\end{equation*}
Se la funzione \`e strettamente crescente o decrescente \`e iniettiva. $f:A \subset \mathbf{R}\rightarrow\mathbf{R}$ \`e iniettiva se ogni retta orizzontale interseca il suo 
grafico al pi\`u in un punto.
\subsection{Suriettività}
Una funzione $f: \mathbb{X} \rightarrow \mathbb{Y}$ si dice \textbf{suriettiva} se:
\begin{equation*}
\forall y \in \mathbb{Y}\ \exists\ x \in \mathbb{X}\ |\ f(x) = y
\end{equation*} 
Analogamente, il concetto di suriettività può essere definito come:
\begin{equation*}
Imf = \mathbb{Y}
\end{equation*}

\subsection{Biettività}
Una funzione $f: \mathbb{X} \rightarrow \mathbb{Y}$ si dice \textbf{biiettiva} se è sia iniettiva che suriettiva.
\begin{equation*}
f: \mathbb{X} \rightarrow \mathbb{Y}\ biettiva \iff \forall\ y \in \mathbb{Y}\ \exists !\ x \in \mathbb{X}\ |\ f(x)=y
\end{equation*}

\section{Grafico}
Sia $f: \mathbb{X} \rightarrow \mathbb{Y}$; si dice grafico di $f$ (indicato con $G(f)$ o $graph(f)$) il sottoinsieme di $\mathbb{X}\times\mathbb{Y}$ definito come:
\begin{equation*}
graph(f)=\{(x,f(x)), x \in \mathbb{X}\} \subseteq \mathbb{X}\times\mathbb{Y}
\end{equation*}
Spesso si usa impropriamente la parola \textit{funzione} per indicare il \textit{grafico di funzione}, ma è generalmente accettato a scopo di semplificazione del discorso.

\section{Altre funzioni particolari}
\subsection{Funzione parte intera}
\begin{equation}
\begin{gathered}
$[$\cdot$]$: \mathbb{R} \rightarrow \mathbb{R}\\
x \rightarrow [x] = max\{n \in \mathbb{Z} | n\leq x\}
\end{gathered}
\end{equation}

\subsection{Funzione di Heaviside}
\begin{equation}
\begin{gathered}
H: \mathbb{R} \rightarrow \mathbb{R}\\
x \rightarrow H(x) =
\begin{cases}
1 \text{  se } x \geq 0\\
0 \text{  se } x < 0
\end{cases}
\end{gathered}
\end{equation}

\subsection{Funzione segno}
\begin{equation}
\begin{gathered}
sgn: \mathbb{R} \rightarrow \mathbb{R}\\
x \rightarrow sgn(x) =
\begin{cases}
1 \text{ se } x > 0\\
0 \text{ se } x = 0\\
-1 \text{ se } x < 0
\end{cases}
\end{gathered}
\end{equation}

\subsection{Funzione mantissa}
\begin{equation}
\begin{gathered}
f: \mathbb{R} \rightarrow \mathbb{R}\\
x \rightarrow f(x) = x - $[$x$]$
\end{gathered}
\end{equation}

\subsection{Funzione parte positiva}
\begin{equation}
\begin{gathered}
f_+: \mathbb{X} \rightarrow \mathbb{R}\\
f_+(x)=\begin{cases}
f(x) \text{ se } x>0\\
0 \text{ se } x \leq 0\\
\end{cases}
\end{gathered}
\end{equation}

\subsection{Funzione parte negativa}
\begin{equation}
\begin{gathered}
f_-: \mathbb{X} \rightarrow \mathbb{R}\\
f_-(x)=\begin{cases}
-f(x) \text{  se } x<0\\
0 \text{  se } x \geq 0\\
\end{cases}
\end{gathered}
\end{equation}

\section{Proprietà}
\subsection{Monotonia}
Sia $\mathbb{A} \subseteq \mathbb{X} \subseteq \mathbb{R}$ e $f: \mathbb{X} \rightarrow \mathbb{R}$ funzione detta (nell'insieme $\mathbb{A}$):
\begin{enumerate}
\item \textbf{crescente}: $\forall\ x_1,x_2 \in \mathbb{A}$, $x_1<x_2$; $f(x_1) \leq f(x_2)$
\item \textbf{strettamente crescente}: $\forall\ x_1,x_2 \in \mathbb{A}$, $x_1<x_2$; $f(x_1) < f(x_2)$
\item \textbf{decrescente}: $\forall\ x_1,x_2 \in \mathbb{A}$, $x_1<x_2$; $f(x_1) \geq f(x_2)$
\item \textbf{strettamente decrescente}: $\forall\ x_1,x_2 \in \mathbb{A}$, $x_1<x_2$; $f(x_1) > f(x_2)$
\item \textbf{monotona}: $f$ crescente oppure decrescente
\item \textbf{strettamente monotona}: $f$ strettamente crescente oppure strettamente decrescente
\end{enumerate}
\subsubsection{Osservazioni}
\begin{enumerate}
\item La somma di due funzioni crescenti è crescente;\\
La somma di due funzioni decrescenti è decrescente
\item Il prodotto di due funzioni crescenti non negative è crescente
\end{enumerate}

\subsection{Parità}
Sia $\mathbb{X}$ \textbf{simmetrico} rispetto l'origine (cioè: $\forall\ x \in \mathbb{X}$; $-x \in \mathbb{X}$); se $f: \mathbb{X} \rightarrow \mathbb{R}$ allora:
\begin{enumerate}
\item[i.] $f$ si dice \textbf{pari} in $\mathbb{X}$ se $f(x) = f(-x)\ \forall\ x \in \mathbb{X}$
\item[ii.] $f$ si dice \textbf{dispari} in $\mathbb{X}$ se $f(x) = -f(-x)\ \forall\ x \in \mathbb{X}$
\end{enumerate}

\subsection{Periodicità}
Sia $\mathbb{X} \subseteq \mathbb{R}$;   $f: \mathbb{X} \rightarrow \mathbb{R}$ si dice \textbf{periodica} di periodo $T>0$ se $T$ è il più piccolo numero reale tale che $x + T \in \mathbb{X}$, $f(x) = f(x + T)$\\
Ogni intervallo di lunghezza $T$ è detto intervallo di periodicità

\section{Estremi di funzione}
Sia $f: \mathbb{X} \rightarrow \mathbb{R}$
\subsection{Positività}
\begin{enumerate}
\item $f$ si dice \textbf{positiva} in $\mathbb{A} \subseteq \mathbb{R} \iff f(x)>0\ \forall\ x \in \mathbb{A}$
\item $f$ si dice \textbf{non negativa} in $\mathbb{A} \subseteq \mathbb{R} \iff f(x) \geq 0\ \forall\ x \in \mathbb{A}$
\item $f$ si dice \textbf{negativa} in $\mathbb{A} \subseteq \mathbb{R} \iff f(x)<0\ \forall\ x \in \mathbb{A}$
\item $f$ si dice \textbf{non positiva} in $\mathbb{A} \subseteq \mathbb{R} \iff f(x) \leq 0\ \forall\ x \in \mathbb{A}$
\end{enumerate}
\subsection{Limitazione, massimo assoluto, minimo assoluto, estremo superiore e inferiore e caratterizzazione}
Gli estremi assoluti di $f$ sono gli stessi dell'insieme $f(\mathbb{X}) = Immf$; per le definizioni fare quindi riferimento agli \hyperref[sec: estremiInsiemi]{\color{cyan}estremi di insiemi}.
\subsection{Massimo relativo e minimo relativo}
TBD
\section{Successioni}
La funzione $f: \mathbb{N} \rightarrow \mathbb{R}$;  $n \rightarrow f(n) = a_n$ è chiamata successione e scritta anche come: $\{a_n\}_{n \in \mathbb{N}}$

\section{Composizione di funzioni}
Siano: $f: domf \rightarrow \mathbb{Y}$; $g: domg \rightarrow \mathbb{W}$ due funzioni e $\mathbb{A}=\{x \in domf\ |\ f(x) \in domg\}$. Si può allora definire:\\
\begin{equation}
\begin{gathered}
(g \circ f)(x) = g(f(x))\\
g \circ f: \mathbb{A} \in \mathbb{W}
\end{gathered}
\end{equation}

In particolare $\mathbb{A} \subseteq \mathbb{Y}$ ed è sempre definita $g \circ f \restriction _\mathbb{A}$
\subsubsection{Osservazione}
$f \circ g \neq g \circ f$

\subsection{Proprietà}
Sia $\mathbb{A} \subseteq dom(g \circ f)$.
\begin{enumerate}
\item[i.] $\begin{cases}
f \text{ crescente in } \mathbb{A}\\
g \text{ crescente in } f(\mathbb{A})
\end{cases} \implies g \circ f \text{ crescente}$
\item[ii.] $\begin{cases}
f \text{ crescente in } \mathbb{A}\\
g \text{ decrescente in } f(\mathbb{A})
\end{cases} \implies g \circ f \text{ decrescente}$
\item[iii.] $\begin{cases}
f \text{ decrescente in } \mathbb{A}\\
g \text{ crescente in } f(\mathbb{A})
\end{cases} \implies g \circ f \text{ decrescente}$
\item[iv.] $\begin{cases}
f \text{ decrescente in } \mathbb{A}\\
g \text{ decrescente in } f(\mathbb{A})
\end{cases} \implies g \circ f \text{ crescente}$
\end{enumerate} 

\section{Funzione inversa}
Sia $f: \mathbb{X} \rightarrow \mathbb{Y}$ una funzione biettiva. La \textbf{funzione inversa} di $f$ si definisce come:
\begin{equation}
\exists\ f^{-1}: \mathbb{Y} \rightarrow \mathbb{X};\ y \rightarrow x = f^{-1}(y)\ |\ f(x) = y
\end{equation}
\subsection{Osservazioni}
\begin{enumerate}
\item $f^{-1}(f(x))=x\\
f(f^{-1}(y))=y$
\item Se $(y,x) \in graph(f^{-1}) \implies f^{-1}(y) = x \implies f(x)=y \implies (x,y) \in graph(f)$\\
(Il grafico di $f$ e di $f^{-1}$ sono simmetrici rispetto la bisettrice)
\end{enumerate}
\subsection{Funzioni goniometriche inverse}
Siano le funzioni goniometriche:
\begin{equation}
\begin{gathered}
sen: [-\frac{\pi}{2};\frac{\pi}{2}] \rightarrow [-1;1]\\
cos: [0;\pi] \rightarrow [-1;1]\\
tan: [-\frac{\pi}{2};\frac{\pi}{2}] \rightarrow \mathbb{R}
\end{gathered}
\end{equation}
Le funzioni inverse sono:
\begin{equation}
\begin{gathered}
arcsen: [-1;1] \rightarrow [-\frac{\pi}{2};\frac{\pi}{2}]\\
arccos: [-1;1] \rightarrow [0;\pi]\\
arctan: \mathbb{R} \rightarrow [-\frac{\pi}{2};\frac{\pi}{2}]
\end{gathered}
\end{equation}