\chapter{Estremi di insiemi}
\label{sec: estremiInsiemi}
\section{Intervalli}
Si definiscono come \textbf{intervalli} alcuni particolari sottoinsiemi di $\mathbb{R}$:
\begin{equation}
\begin{gathered}
$[a, b$] = \{x\in\mathbb{R}: a\le x\le b\ \} \text{ intervallo \textbf{chiuso}}\\
$(a, b$) = \{x\in\mathbb{R}: a < x < b\} \text{ intervallo \textbf{aperto}}\\
$(-$\infty$, b$] = \{x\in\mathbb{R}: x\le b\} \text{ intervallo \textbf{illimitato}}\\
$[a, +$\infty$$) = \{x\in\mathbb{R}: x\ge a\} \text{ intervallo \textbf{illimitato}}\\
\end{gathered}
\end{equation}
\section{Maggiorante}
Sia $\mathbb{A} \subseteq \mathbb{R}$. Si dice che $\mathbb{A}$ è \textbf{superiormente limitato} se:
\begin{equation}
\exists\ M \in \mathbb{R}\ |\ x \leq M\ \forall\ x \in \mathbb{A}
\end{equation}
$M$ viene chiamato \textbf{maggiorante}.
\section{Minorante}
Sia $\mathbb{A} \subseteq \mathbb{R}$. Si dice che $\mathbb{A}$ è \textbf{inferiormente limitato} se:
\begin{equation}
\exists\ m \in \mathbb{R}\ |\ x \geq m\ \forall\ x \in \mathbb{A}
\end{equation}
$m$ viene chiamato \textbf{minorante}.
\section{Massimo}
Sia $\mathbb{A} \subseteq \mathbb{R}$. $max\mathbb{A}=\bar{x} \in \mathbb{R}$ è \textbf{massimo} di $\mathbb{A}$ se:
\begin{equation}
\begin{cases}
x \leq \bar{x}\ \forall\ x \in \mathbb{A}\\
\bar{x} \in \mathbb{A}
\end{cases}
\end{equation}
\section{Minimo}
Sia $\mathbb{A} \subseteq \mathbb{R}$. $min\mathbb{A}=\hat{x} \in \mathbb{R}$ è \textbf{minimo} di $\mathbb{A}$ se:
\begin{equation}
\begin{cases}
x \geq \hat{x}\ \forall\ x \in \mathbb{A}\\
\hat{x} \in \mathbb{A}
\end{cases}
\end{equation}
\subsection{Osservazioni circa massimo/minimo}
\begin{enumerate}
\item [i.] Se $\exists\ max\mathbb{A}$ o $\exists\ min\mathbb{A}$, essi sono \textbf{unici}.
\item [ii.] $\exists\ max\mathbb{A} \iff \exists\ M \in \mathbb{R}\ |\ x \leq M\ \forall\ x \in \mathbb{A}\\
\exists\ min\mathbb{A} \iff \exists\ m \in \mathbb{R}\ |\ x \geq m\ \forall\ x \in \mathbb{A}$
\end{enumerate}
\section{Estremo superiore}
Sia $\mathbb{A} \subseteq \mathbb{R}$. Viene definito \textbf{estremo superiore} di A:
\begin{equation}
sup\mathbb{A} = min\{M \in \mathbb{R} | x \leq M\ \forall\ x \in \mathbb{A}\}
\end{equation}
\subsubsection{Caratterizzazione}
Sia $\mathbb{A} \subset \mathbb{R}$ superiormente limitato, $s \in \mathbb{R}$:\\
\begin{equation}
	s=sup\mathbb{A} \iff
	\begin{cases}
	s \geq x\ \forall\ x \in \mathbb{A}\\
	\forall\ \epsilon > 0\ \exists\ x \in \mathbb{A}\ |\ x > s-\epsilon
	\end{cases}
\end{equation}
\section{Estremo inferiore}
Sia $\mathbb{A} \subseteq \mathbb{R}$. Viene definito \textbf{estremo inferiore} di A:
\begin{equation}
inf\mathbb{A} = max\{m \in \mathbb{R} | x \geq m\ \forall\ x \in \mathbb{A}\}
\end{equation}
\subsubsection{Caratterizzazione inf}
Sia $\mathbb{A} \subset \mathbb{R}$ inferiormente limitato, $t \in \mathbb{R}$:\\
\begin{equation}
	t=inf\mathbb{A} \iff
	\begin{cases}
	t \leq x\ \forall\ x \in \mathbb{A}\\
	\forall\ \epsilon > 0\ \exists\ x \in \mathbb{A}\ |\ x < t+\epsilon
	\end{cases}
\end{equation}

\section{Osservazioni estermo superiore/inferiore}
\begin{enumerate}
\item Se $\exists\ max\mathbb{A} \implies \exists\ supA=max\mathbb{A}\\
$Se $\exists\ minA \implies \exists\ infA=min\mathbb{A}$
\item Se $\exists\ sup\mathbb{A}, sup\mathbb{A} \in \mathbb{A} \implies \exists max\mathbb{A} = sup\mathbb{A}$\\
Se $\exists\ inf\mathbb{A}, inf\mathbb{A} \in \mathbb{A} \implies \exists min\mathbb{A} = inf\mathbb{A}$
\end{enumerate}

\section{Insiemi finiti}
Un insieme $\mathbb{A}$ si dice \textbf{finito} se ha un numero finito di elementi.\\
Se $\mathbb{A} \subset \mathbb{R}$, $\mathbb{A}$ finito, $\mathbb{A} \neq \emptyset \implies \exists\ max\mathbb{A}, min\mathbb{A}$

\label{sec: CompletezzaReali}
\section{Completezza dell'insieme dei Reali}
Sia $\mathbb{A} \subset \mathbb{R}$, $\mathbb{A} \neq \emptyset$:
\begin{center}
Se $\mathbb{A}$ è limitato superiormente (o inferiormente) $\implies \exists\ sup\mathbb{A} \in \mathbb{R}$ (o $\exists\ inf\mathbb{A} \in \mathbb{R}$)
\end{center}